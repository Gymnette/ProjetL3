\documentclass[a4paper,12pt]{article}

\usepackage{framed}
\usepackage[utf8]{inputenc}
\usepackage[T1]{fontenc}
\usepackage[french]{babel} 
\usepackage{lmodern} % Pour changer le pack de police
\usepackage{makeidx}
\usepackage{graphicx}
\usepackage{wrapfig}
\usepackage{amsmath}
\usepackage{amssymb}
\usepackage{bclogo}
\usepackage[top=2cm, bottom=3cm, left=2.5cm, right=2.5cm]{geometry} %Marges

\renewcommand{\baselinestretch}{1.3} %Interligne

\newcommand{\tend}[2]{\displaystyle\mathop{\longrightarrow}_{#1\rightarrow#2}} 
\newcommand{\IKex}[2]{\mathbb{#1}^{#2}}
\newcommand{\IK}[1]{\mathbb{#1}}
\newcommand{\IRex}[1]{\IKex{R}{#1}}
\newcommand{\IR}{\IK{R}}
\newcommand{\dans}[2]{#1 \rightarrow #2}

\title{Cahier des Charges}
\author{\textbf{Interpolaspline}\\Stage Applicatif L3 MIN}
\date{Année 2019-2020}

\begin{document}

\maketitle
\tableofcontents

\newpage

\section{Contexte et définition du problème}

\subsection{Motivations}

Lors d'expériences scientifiques, des données sont acquises. Cependant, ces données sont très souvent discrètes ce qui ne permet pas d'établir facilement une loi (fonction de $\IR$ dans $\IRex{n}$ suivie par ces données) continue. C'est le principe de l'interpolation.

Cependant, dans la majorité des expériences, une fonction passant par toutes les données oscillerait trop. On trouve donc une fonction minimisant une certaine quantité, liée à la distance entre les données et la fonction établie. Dans ce cas, on parle d'approximation.

Lors des expériences, si une donnée est très éloignée des autres, on peut souhaiter l'ignorer. Cela peut-être une erreur de mesure, un changement de conditions, une grosse imprécision, etc. Ces données sont dites aberrantes, et c'est ce cas qui nous intéresse dans ce projet.
\\ \\
Voici quelques exemples d'utilisation de l'interpolation et de l'approximation :
\begin{itemize}
\item Étalonnage d'instruments de mesure, en métrologie
\item Découverte de lois à partir d'expériences, dans tous les domaines de la recherche
\item Identifier des données incohérentes : une faute de frappe, un produit défectueux
\end{itemize}


\subsection{Problème}

[La base de l'interpolation est de trouver une loi qui coïncide parfaitement avec les données. Cependant, les données expérimentales peuvent imprécises, et peuvent ne pas recouper exactement la loi.]

L’interpolation consiste à remplacer une courbe ou une fonction par une autre courbe (ou fonction) plus simple, mais qui coïncide avec la première en un nombre fini de points (ou de valeurs) donnés au départ. Suivant la méthode que l’on aura choisie, on peut demander à la courbe ou la fonction de vérifier certaines propriétés en plus. Le choix des points (ou valeurs) au départ est un élément important dans la construction.
Ces données expérimentales peuvent être imprécises. L’idée c’est de trouver une loi qui coïncide parfaitement avec les données.

Les fonctions d’interpolation utilisées ici sont des splines.
Une spline d’interpolation est une fonction par morceaux constituée d’un polynôme sur chaque intervalle.
L’idée c’est de combiner des fonctions simples (typiquement des polynômes de bas degré) de manière à former une fonction composite lisse (c’est-à-dire de classe Ck).

L’évaluation des paramètres de la spline peuvent diverger si la pente entre deux nœuds est beaucoup plus grande que 1, la spline dépend du choix du système de coordonnées, elle ne possède donc pas de propriété d’invariance géométrique, on ne peut pas l’utiliser si la courbe à représenter n’est pas une fonction, c’est-à-dire s’il y’a plus d’une abscisse donnée (courbes qui « reviennent en arrière » comme les ellipses, les cercles, etc.)

Pour approcher ces données parfois imprécises on utilise les splines de lissage. L’objectif du lissage est de fournir une estimation non bruitée des données avec un paramètre de lissage rho qui permet de contrôler le compromis entre les données observées et le lissage de la fonction.

\newpage
\section{Objectifs}

Une spline naturelle est une fonction définie par morceau par des polynômes cubiques, dont la dérivée seconde est continue, et dont les dérivées secondes aux extrémités de l'intervalle de définition sont nulles.
Le but du projet est de pouvoir créer une spline naturelle approximant un ensemble de données contenant des points aberrants, à une dimension puis à deux dimensions (splines naturelles paramétriques).

\subsection*{Contenu}
Le projet sera livré sous la forme d'une archive contenant :
\begin{itemize}
\item[•] Plusieurs fichiers python regroupant les différentes fonctionnalités décrites à la section \ref{Fonctions}  : 
\begin{itemize}
\item Une librairie "Splines 1D" (nommée $"Splines1D.py"$), qui regroupe les méthodes de création des splines à une dimension
\item Une librairie "Splines Paramétriques" (nommée $"SplinesParametriques.py"$), qui regroupe les méthodes de création des splines paramétriques
\item Une librairie "Données Aberrantes" (nommée $"DonneesAberrantes.py"$), qui regroupe les méthodes de gestion des points aberrants
\item Une librairie "Automatisation" (nommée $"Automatisation.py"$), qui regroupe les méthodes de configuration automatique des paramètres des splines
\item Un fichier "Interface" (nommé $"Interface.py"$), qui contient le programme principal, qui est le seul que l'utilisateur doit lancer.
\end{itemize}
\item[•] Un lisez-moi consignant les dépendances entre les programmes, et les instructions d'utilisation.
\item[•] Un rapport expliquant les diverses méthodes, avec des exemples d’utilisations illustrés.
\end{itemize}

\newpage
\section{Limites}

Restriction de la solution proposée (cadrage) :
\begin{itemize}
\item Les points donnés en entrée seront défini dans un ordre précis.
\end{itemize}

\newpage
\section{Description fonctionnelle}

\subsection{Fonctionnalités}
\label{Fonctions}


\subsection{Entrées et sorties}

Input : fichiers textes contenant les données

Output : fichier contenant les valeurs aberrantes
	   Fichier contenant une discrétisation de la foncti on résultat
	   Affichage de la fonction résultat

\newpage
\section{Organisation temporelle}

Dates importantes et livraison :
\begin{itemize}
\item Mercredi 18 décembre : présentation du cahier des charges et de l'organisation du projet.
\item Jeudi 30 avril : soutenance du projet et rendu des codes.
\end{itemize}



\end{document}