\documentclass[a4paper,12pt]{article}

\usepackage{framed}
\usepackage[utf8]{inputenc}
\usepackage[T1]{fontenc}
\usepackage[french]{babel} 
\usepackage{lmodern} % Pour changer le pack de police
\usepackage{makeidx}
\usepackage{graphicx}
\usepackage{wrapfig}
\usepackage{amsmath}
\usepackage{amssymb}
\usepackage{bclogo}
\usepackage[top=2cm, bottom=3cm, left=2.5cm, right=2.5cm]{geometry} %Marges

\renewcommand{\baselinestretch}{1.3} %Interligne

\newcommand{\tend}[2]{\displaystyle\mathop{\longrightarrow}_{#1\rightarrow#2}} 
\newcommand{\IKex}[2]{\mathbb{#1}^{#2}}
\newcommand{\IK}[1]{\mathbb{#1}}
\newcommand{\IRex}[1]{\IKex{R}{#1}}
\newcommand{\IR}{\IK{R}}
\newcommand{\dans}[2]{#1 \rightarrow #2}

\title{Cahier des Charges}
\author{L3}
\date{Annee 2019-2020}

\begin{document}

\maketitle
\tableofcontents

\newpage

\section{Contexte et définition du problème}

\subsection{Motivations}

Lors d'expériences scientifiques, des données sont acquises. Cependant, ces données sont très souvent discrètes, et ne permettent pas de tirer des conclusions telles quelles. Il faut ainsi trouver une loi (une fonction pour des données de $\IR$ dans $\IRex{n}$) que suivent ces données. C'est le principe de l'interpolation.
\\ \\
Comme exemples d'utilisations, on peut citer :
\begin{itemize}
\item Étalonnage d'instruments de mesure, en métrologie
\item Découverte de lois à partir d'expériences, dans tous les domaines de la recherche
\item Identifier des données incohérentes : une faute de frappe, un produit défectueux
\end{itemize}

\subsection{Problème}

La base de l'interpolation est de trouver une loi qui coïncide parfaitement avec les données. Cependant, les données expérimentales peuvent imprécises, et peuvent ne pas recouper exactement la loi. 

\newpage
\section{Objectif}

Le projet sera livré sous la forme d'une archive contenant :
\begin{itemize}
\item[•] Plusieurs packages python regroupant les différentes foncitonnalités décrite à la section \ref{Fonctions}  : 
\begin{itemize}
\item Package "Splines 1D", qui regroupe les méthodes de création des splines 1D.
\item Package "Splines Paramétriques", qui regroupe les méthodes de création des splines paramétriques.
\item Package "Données Aberrantes" qui regroupe les méthodes de gestion des points aberrants.
\item Package "Automatisation" qui regroupe les méthodes de configuration automatique des paramètres.
\item Package "Interface" qui contient le programme principa à lancer par l'utilisateur.
\end{itemize}
\item[•] Un read-me conciliant les dépendances entre les programmes, et les instructions d'utilisation.
\item[•] Rapport d'explications des divers méthodes, avec exemples d’utilisations sur des jeux de données, illustrés.
\end{itemize}

\newpage
\section{Périmètre}

Restriction de la solution proposée (cadrage) :
\begin{itemize}
\item Les points donnés en entrée seront défini dans un ordre précis.
\end{itemize}

\newpage
\section{Description fonctionnelle}

\subsection{Fonctionnalités}
\label{Fonctions}


\subsection{Entrées et Sorties}

Input : fichiers textes contenant les données

Output : fichier contenant les valeurs aberrantes
	   Fichier contenant une discrétisation de la foncti on résultat
	   Affichage de la fonction résultat

\newpage
\section{Délais Livrables}

échéancier, dates importantes, livraison :
\begin{itemize}
\item Mercredi 11 décembre : présentation du cahier des charges.
\item Jeudi 30 avril : soutenance du projet et rendu des codes.
\end{itemize}



\end{document}