\documentclass[a4paper,12pt]{article}

\usepackage{framed}
\usepackage[utf8]{inputenc}
\usepackage[T1]{fontenc}
\usepackage[french]{babel} 
\usepackage{lmodern} % Pour changer le pack de police
\usepackage{makeidx}
\usepackage{graphicx}
\usepackage{wrapfig}
\usepackage{amsmath}
\usepackage{amssymb}
\usepackage{bclogo}
\usepackage[top=2cm, bottom=3cm, left=2.5cm, right=2.5cm]{geometry} %Marges

\renewcommand{\baselinestretch}{1.3} %Interligne

\newcommand{\tend}[2]{\displaystyle\mathop{\longrightarrow}_{#1\rightarrow#2}} 
\newcommand{\IKex}[2]{\mathbb{#1}^{#2}}
\newcommand{\IK}[1]{\mathbb{#1}}
\newcommand{\IRex}[1]{\IKex{R}{#1}}
\newcommand{\IR}{\IK{R}}
\newcommand{\dans}[2]{#1 \rightarrow #2}

\title{Cahier des Charges}
\author{L3}
\date{Annee 2019-2020}

\begin{document}

\maketitle
\tableofcontents

\newpage

\section{Contexte et définition du problème}

\subsection{Motivations}

Lors d'expériences scientifiques, des données sont acquises. Cependant, ces données sont très souvent discrètes, et ne permettent pas de tirer des conclusions telles quelles. Il faut ainsi trouver une loi (une fonction pour des données de $\IR$ dans $\IRex{n}$) que suivent ces données. C'est le principe de l'interpolation.
\\ \\
Comme exemples d'utilisations, on peut citer :
\begin{itemize}
\item Étalonnage d'instruments de mesure, en métrologie
\item Découverte de lois à partir d'expériences, dans tous les domaines de la recherche
\item Identifier des données incohérentes : une faute de frappe, un produit défectueux
\end{itemize}

\subsection{Problème}

[La base de l'interpolation est de trouver une loi qui coïncide parfaitement avec les données. Cependant, les données expérimentales peuvent imprécises, et peuvent ne pas recouper exactement la loi.]

L’interpolation consiste à remplacer une courbe ou une fonction par une autre courbe (ou fonction) plus simple, mais qui coïncide avec la première en un nombre fini de points (ou de valeurs) donnés au départ. Suivant la méthode que l’on aura choisie, on peut demander à la courbe ou la fonction de vérifier certaines propriétés en plus. Le choix des points (ou valeurs) au départ est un élément important dans la construction.
Ces données expérimentales peuvent être imprécises. L’idée c’est de trouver une loi qui coïncide parfaitement avec les données.

Les fonctions d’interpolation utilisées ici sont des splines.
Une spline d’interpolation est une fonction par morceaux constituée d’un polynôme sur chaque intervalle.
L’idée c’est de combiner des fonctions simples (typiquement des polynômes de bas degré) de manière à former une fonction composite lisse (c’est-à-dire de classe Ck).

L’évaluation des paramètres de la spline peuvent diverger si la pente entre deux nœuds est beaucoup plus grande que 1, la spline dépend du choix du système de coordonnées, elle ne possède donc pas de propriété d’invariance géométrique, on ne peut pas l’utiliser si la courbe à représenter n’est pas une fonction, c’est-à-dire s’il y’a plus d’une abscisse donnée (courbes qui « reviennent en arrière » comme les ellipses, les cercles, etc.)

Pour approcher ces données parfois imprécises on utilise les splines de lissage. L’objectif du lissage est de fournir une estimation non bruitée des données avec un paramètre de lissage rho qui permet de contrôler le compromis entre les données observées et le lissage de la fonction.

\newpage
\section{Objectif}

Le but du projet est de s'adapter à un ensemble de données en 2D et en 3D

\subsection*{Contenu}
Le projet sera livré sous la forme d'une archive contenant :
\begin{itemize}
\item[•] Plusieurs packages python regroupant les différentes foncitonnalités décrite à la section \ref{Fonctions}  : 
\begin{itemize}
\item Package "Splines 1D", qui regroupe les méthodes de création des splines 1D.
\item Package "Splines Paramétriques", qui regroupe les méthodes de création des splines paramétriques.
\item Package "Données Aberrantes" qui regroupe les méthodes de gestion des points aberrants.
\item Package "Automatisation" qui regroupe les méthodes de configuration automatique des paramètres.
\item Package "Interface" qui contient le programme principa à lancer par l'utilisateur.
\end{itemize}
\item[•] Un read-me conciliant les dépendances entre les programmes, et les instructions d'utilisation.
\item[•] Rapport d'explications des divers méthodes, avec exemples d’utilisations sur des jeux de données, illustrés.
\end{itemize}

\newpage
\section{Périmètre}

Restriction de la solution proposée (cadrage) :
\begin{itemize}
\item Les points donnés en entrée seront défini dans un ordre précis.
\end{itemize}

\newpage
\section{Description fonctionnelle}

\subsection{Fonctionnalités}
\label{Fonctions}


\subsection{Entrées et Sorties}

Input : fichiers textes contenant les données

Output : fichier contenant les valeurs aberrantes
	   Fichier contenant une discrétisation de la foncti on résultat
	   Affichage de la fonction résultat

\newpage
\section{Délais Livrables}

échéancier, dates importantes, livraison :
\begin{itemize}
\item Mercredi 11 décembre : présentation du cahier des charges.
\item Jeudi 30 avril : soutenance du projet et rendu des codes.
\end{itemize}



\end{document}