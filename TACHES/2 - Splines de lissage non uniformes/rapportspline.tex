\documentclass[11pt]{report}
\usepackage[utf8]{inputenc}
\usepackage[T1]{fontenc}
\usepackage[french]{babel}
\usepackage{geometry}
\usepackage{tikz} 
\usepackage{amsmath}
\usepackage{amssymb}
\usepackage{ifthen}
\usepackage{fancybox}
\usepackage{fancyhdr}
\geometry{left=1cm , right=1cm , top=1.5 cm , bottom=1.5cm}
\usepackage{color}
\usepackage{soul}
\usepackage{graphicx}
\thispagestyle{empty}
\begin{document}


\chapter*{Spline de lissage non-uniformes:}
une spline lissante permet de satisfaire un compromis entre la fidélité aux observations bruyantes et le lissage de la spline ajustée. précisément, étant donné un ensemble de points de données $(u_k,z_k)$ avec $$u_{1}<u_{2}< ... <u_{n}$$
où les observations $z_k$ sont supposées bruyantes, nous considérons une séquence de nœuds splines $x_{1}< x_{2}<... < x_{n}$
tel que $\{{u_k}\}_{1 \leq k \leq N}$, et l'espace $S[x_1,x_2]$ des splines naturelles associées à ces points \\
on considère ensuite le problème d'optimisation $$Min_{s \in S[x_1,x_2]}E_{0,2}(s)$$
$$E_{0,2}(s)=\sum_{k=1}^{N}(z_k-s(u_k))^2+\rho\int_{x_1}^{x_n}[s^{''}(t)]^2$$
de la même maniéré que dans le cas uniforme, on détermine les matrice A,R,S,M,N,$H_{0,3} et H_{1,2}$\\
commençant par la matrice A et R on cherche toujours la relation entre y et y' avec les conditions normal de  spline 
$(s''(x_1)=s''x_n)=0$ et la condition du cantacte $C^2$ aux nœuds internes($s''_{i-1}(x_i)=s''_i(x_i)$\\
à la fin on obtient ces 3 relation entre y et y'\\
$2y'_1+y'_2=3/h_1(y_2-y_1)$\\
$y'_{n-1}+2y'_n=3/h_{n-1}(y_n-y_{n-1})$\\
donc la relation entre y et y' est de la forme Ay'=Ry ,alors on obtient \\
$$A=\begin{pmatrix} 2&1&0&0&...&...&...&0&0 \\ h_2&2(h_1+h_2)&h_1&0&...&...&...&0&0 \\ \vdots&.&.&.&.&...&.&.&.\\ \vdots&.&.&.&h_i&2(h_{i-1}+h_i)&h_{i-1}&0&\vdots \\. \\.&.&.&.&.&...&.&.&. \\ 0&0&.&.&.&...&.&1&2 \end{pmatrix}$$\\

$$R=1/3\begin{pmatrix} -\frac{1}{h_1}&\frac{1}{h_1}&0&0&...&...&...&0&0 \\ \frac{h_2}{h_1}&\frac{h_2}{h_1}-\frac{h_1}{h_2}&\frac{h_1}{h_2}&0&...&...&...&0&0 \\ \vdots&.&.&.&.&...&.&.&.\\ 0&.&.&.&\frac{-h_i}{h_{i-1}}&\frac{h_i}{h_{i-1}}-\frac{h_{i-1}}{h_i}&\frac{h_{i-1}}{h_i}&\vdots&0 \\.&.&.&.&.&...&.&.&. \\ 0&0&.&.&.&...&.&-1/h_{n-1}&1/h_{n-1} \end{pmatrix}$$\newpage
pour trouver la matrice S on considéré  l'intégrale suivant \\
$$\int_{x1}^{xn}[s''(t)]^2dt=\sum_{i=1}^{n-1}\int_{x_i}^{x_{i+1}}[s''(t)]^2dt$$
on développant en remplaçant h par $h_i$   cette formule on obtient une sorte de forme quadratique  $$\int_{x1}^{xn}[s''(t)]^2dt= Y''^T S y"$$
et enfin la matrice S est de la forme 
$$S=1/3\begin{pmatrix} 2h_1&\frac{1}{2}h_1&0&0&...&...&...&0&0 \\ \frac{1}{2}h_2&2h_2&\frac{1}{2}h_2&0&...&...&...&0&0 \\ \vdots&.&.&.&.&...&.&.&.\\ \vdots&.&.&.&\frac{1}{2}h_i&2h_i&\frac{1}{2}h_i&0&\vdots \\. \\.&.&.&.&.&...&.&.&. \\ 0&0&.&.&.&...&.&\frac{1}{2}h_{n-1}&2h_{n-1} \end{pmatrix}$$
pour obtenir les matrice M et N nous exprimons le vecteur y" en fonction des  vecteurs y et y' .Pour les splines naturelles s $\in$ S, avec des segments $s_i=S_{[x_i,x_{i+1}]}$,i=1,..,n-1.\\
enfin on applique l'interpolation d'Hermite avec a les conditions suivantes:$$y"_i=s"_{i-1}(x_i)=s"_i(x_i) ,i=2,..n-1$$
qui mène à $$y"=My+Ny'$$ Donc :
$$M=3\begin{pmatrix} \frac{1}{h_1^2}&-\frac{1}{h_1^2}-\frac{1}{h_2^2}&\frac{1}{h_2^2}&0&0&...&...&...&0&0 \\ \vdots&...&\vdots&.&...&...&...&.&\vdots \\ \vdots&.&.&.&.&...&.&.&.\\ \vdots&.&.&.&\frac{1}{h_{i-1}^2}&-\frac{1}{h_{i-1}^2}-\frac{1}{h_i^2}&\frac{1}{h_i^2}&0&\vdots \\. \\.&.&.&.&.&...&.&.&. \\ 0&0&.&.&.&...&\frac{1}{h_{n-2}^2}&-\frac{1}{h_{n-2}^2}-\frac{1}{h_{n-1}^2}&\frac{1}{h_{n-1}^2} \end{pmatrix}$$
\\
\\$$N=\begin{pmatrix} \frac{1}{h_1}&\frac{2}{h_1}-\frac{2}{h_2}&-\frac{1}{h_2}&0&0&...&...&...&0&0 \\ \vdots&...&\vdots&.&...&...&...&.&\vdots \\ \vdots&.&.&.&.&...&.&.&.\\ \vdots&.&.&.&\frac{1}{h_{i-1}}&\frac{2}{h_{i-1}}-\frac{2}{h_i}&-\frac{1}{h_i}&0&\vdots \\. \\.&.&.&.&.&...&.&.&. \\ 0&0&.&.&.&...&\frac{1}{h_{n-2}}&\frac{2}{h_{n-2}}-\frac{2}{h_{n-1}}&-\frac{1}{h_{n-1}} \end{pmatrix}$$\\
avec la même méthode dans le cas Uniforme on trouve les matrice $H_{03} et H_{12}$
$$H_{03}=\begin{pmatrix} H_0(t_1^1)&H_3(t_1^1)& & & & & & & &  \\ \vdots&\vdots& & & & & & &  \\ H_0(t_{k_1}^1)&H_3(t_k1^1)& & & & & & & \\ &H_0(t_{{k_1}+1}^2)&H_3(t_{{k_1}+1}^2)& & & & & &   \\ &\vdots&\vdots& & & & & &  \\ &H_0(t_{k_2}^2)&H_3(t_{k_2}^2)& & & & & & \\ &\ddots&\ddots & & & & & & & \\ & & & & & & &H_0(t_{{k_{n-2}}+1}^{n-1}) &H_3(t_{{k_{n-2}}+1}^{n-1})\\ & & & & & & &\vdots &\vdots\\ & & & & & & &H_0(t_N^{n-1}) &H_3(t_N^{n-1}) \end{pmatrix}$$


$$H_{12}=\begin{pmatrix} h_1H_1(t_1^1)&h_1H_2(t_1^1)& & & & & & & &  \\ \vdots&\vdots& & & & & & &  \\ h_1H_1(t_{k_1}^1)&h_1H_2(t_k1^1)& & & & & & & \\ &h_2H_1(t_{{k_1}+1}^2)&h_2H_2(t_{{k_1}+1}^2)& & & & & &   \\ &\vdots&\vdots& & & & & &  \\ &h_2H_1(t_{k_2}^2)&h_2H_2(t_{k_2}^2)& & & & & & \\ &\ddots&\ddots & & & & & & & \\ & & & & & & &h_{n-1}H_1(t_{{k_{n-2}}+1}^{n-1}) &h_{n-1}H_2(t_{{k_{n-2}}+1}^{n-1})\\ & & & & & & &\vdots &\vdots\\ & & & & & & &h_{n-1}H_1(t_N^{n-1}) &h_{n-1}H_2(t_N^{n-1}) \end{pmatrix}$$


\end{document}